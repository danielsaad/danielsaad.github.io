\documentclass{article}
\usepackage{fullpage}
\usepackage[utf8]{inputenc}
\usepackage[brazilian]{babel}
\usepackage{algorithm2e}
\usepackage{amsmath,amssymb,amsthm}
\usepackage{graphicx}
\usepackage{natbib}
\usepackage{bibentry}
\usepackage{url}
\usepackage{hyperref}
\author{Prof. Daniel Saad Nogueira Nunes}
\title{Ideias de Trabalhos de Conclusão de Curso}
\date{}


\newcommand*{\nsubsection}[1]{
    \subsection*{#1}
}


\newcommand{\ds}{\texttt{DS-Contest-Tools}\xspace}
\newcommand{\polygon}{\texttt{Polygon}\xspace}
\newcommand{\codeforces}{\texttt{Codeforces}\xspace}
\newcommand{\sqtpm}{\texttt{SQTPM}\xspace}
\newcommand{\boca}{\texttt{BOCA}\xspace}
\newcommand{\pip}{\texttt{PIP}\xspace}
\newcommand{\cdmoj}{\texttt{CD-MOJ}\xspace}
\begin{document}
\maketitle
\setcounter{tocdepth}{2}
\setcounter{secnumdepth}{2}
\tableofcontents
\nobibliography{bibliografia.bib}	
\bibliographystyle{plainnat}

\newpage
% \nsubsection{Adequação dos modelos \LaTeX{} para escrita de trabalhos de conclusão de curso}

% Os modelos \LaTeX{} para escrita de conclusão de curso possuem uma série de problemas em relação as normas descritas no \href{https://normaliza.ifb.edu.br/doku.php}{site do IFB} e a norma da ABNT. A adequação desses modelos deve ser realizada com respeito à essas normas.

\section{Ideias para TCC}

\nsubsection{Implementação do suporte do juiz online CD-MOJ à plataforma Codeforces}

De acordo com, o Contest-Driven Meta Online Judge (CD-MOJ) é um juiz online um
online judge direcionado a contests que despacha os códigos para outro juizes
online. Atualmente ele fornece suporte ao próprio banco de questões e ao
SPOJ-BR. A integração do CD-MOJ ao banco de questões do Codeforces é desejável, dado que se trata do
maior portal de programação competitival atualmente.



\subsubsection*{Referências}

\begin{itemize}
	\item \bibentry{cdmojsite}
	\item \bibentry{cdmoj1}
	\item \bibentry{cdmoj2}
\end{itemize}

\nsubsection{Aplicação de algoritmos bioinspirados no problema {\tt BIN-PACKING}}

O problema {\tt BIN-PACKING} é sabidamente um problema $\mathcal{N\!P}$-difícil. Ele consiste em, dado uma coleção de $n$ itens com pesos $W=(w_1,\ldots,w_n)$ e uma capacidade de pacote $C$, determinar o número mínimo de pacotes que podem empacotar os itens, sem que o somatório dos pesos dos itens inseridos em capa pacote não exceda $C$. Uma forma de resolver este problema em tempo viável com uma qualidade de solução satisfatória é utilizar metaheurísticas bioinspiradas. Este problema visa a aplicação destas metaheurísticas e comparação com algoritmos aproximado.


\subsubsection*{Referências}

\begin{itemize}
	\item \bibentry{man1996approximation}
	\item \bibentry{munien2021metaheuristic}
\end{itemize}



\nsubsection{Estudo do problema de alocação de horários nos cursos de computação do IFB-Taguatinga}

O problema de alocação de horários consiste em determinar os horários das disciplinas dos cursos de computação de modo a minimizar os conflitos de horários em relação aos estudantes e outras disciplinas. Este problema não possui estudos do ponto de vista computacional, e sua modelagem e resolução pode ajudar na construção de um horário que cause menos problemas. A investigação do problema em relação à sua classe de complexidade computacional também é interessante. 

\subsubsection*{Referências}


\begin{itemize}
	\item \bibentry{hillier2013introduccao}
\end{itemize}


% \nsubsection{Criação de uma sistema para correção de exercícios de programação}

% Este trabalho visa a criação de um sistema para correção de trabalhos de programação, possibilitando que o aluno obtenha feedback instantâneo durante a execução dos exercícios. Este sistema também poderá dar suporte à atividades avaliativas de programação, facilitando a correção dos códigos-fonte pelo prof. da disciplina. O sistema {\tt SQTPM} pode ser usado como base e estendido.

% \subsubsection*{Referências}


% \begin{itemize}
% 	\item \bibentry{sqtpm}
% \end{itemize}




\nsubsection{Estudo de ataques ao Datacenter do IFB Taguatinga}

O Datacenter do IFB Taguatinga frequentemente sofre ataques de diversos tipos, como: DDoS, ataques de dicionário, interpretação de URLs, dentre outros.
Esse estudo visa produzir um relatório de ataques ao Datacenter através da observação dos mesmos em um \textit{honeypot}. Além disso, possíveis contramedidas serão propostas a cada um dos ataques para que sejam implantadas no Datacenter.

\subsubsection*{Referências}

\begin{itemize}
	\item \bibentry{bringer2012survey}
	\item \bibentry{javadpour2024comprehensive}
	\item \bibentry{ilg2023survey}
	\item \bibentry{stallings2015computer}
\end{itemize}



\nsubsection{Elaboração de roteiros de computação desplugada}

A computação desplugada serve como mecanismo para ensino da computação e do pensamento computacional.Diversos roteiros disponíveis contemplam os mais variados assuntos acerca da Ciência da Computação, mas é possível contribuir com novos roteiros didáticos. Este trabalho propõe a criação de roteiros didáticos para aplicação da computação desplugada nas escolas.

\subsubsection*{Referências}

\begin{itemize}
	\item \bibentry{csunplugged}
\end{itemize}

\newpage

\section{Orientações em andamento}



\nsubsection{Criação de um sistema de consulta de exercícios de programação}

\begin{itemize}
	\item Aluno: Fábio Henrique Lapa
\end{itemize}

Devido ao alto volume de exercícios de programação elaborados para os eventos de programação competitiva no DF, um sistema que classificasse esses problemas de acordo com categorias e permitisse consulta seria interessante para ajudar na capacitação desses estudantes para os eventos supracitados.



\subsubsection*{Referências}

\begin{itemize}
	\item \bibentry{site-maratona-ifb}
\end{itemize}


\nsubsection{Evolução da ferramenta de formatação de problemas DS ContestTools}


\begin{itemize}
	\item Aluno: Leonam Knupp
\end{itemize}


Uma ferramenta de formatação de problemas para juízes eletrônicos possibilita ao autor do problema escrever e formatar o enunciado do problema; gerar os casos de teste; validar os casos de teste; testar as soluções esperadas;  gerar o pacote para o juiz eletrônico escolhido; além de outras funcionalidades.

A ferramenta de formatação de problemas  inicialmente foi proposta para formatar problemas computacionais para a plataforma de formatação de competições de programação \polygon, do \codeforces. Com o tempo ela foi sendo aprimorada, possibilitando vários recursos como:

\begin{itemize}
	\item Conversão para os sistemas \sqtpm e \boca.
	\item Produção de um relatório gráfico dos tempos de execução e memória utilizada pelas soluções esperadas.
	\item Utilização, através da ferramenta, da API do \polygon para envio de problemas ao \codeforces, automatizando o processo de formatação de problemas nessa plataforma.
	\item Documentação em site próprio.
\end{itemize}

Apesar de possuir diversas funcionalidades, ainda há margem para melhorias, como:

\begin{itemize}
	\item Disponibilização da ferramenta no repositório \pip da linguagem de programação Python.
	\item Conversão dos pacotes do problema para a plataforma \cdmoj.
	\item Customização do formato do arquivo PDF gerado por cada problema a depender do tipo de evento ao qual o problema foi pensado.
	\item Entre outras.
\end{itemize}


\subsubsection*{Referências}

\begin{itemize}
	\item \bibentry{github-ds-contest-tools}
	\item \bibentry{site-ds-contest-tools}
\end{itemize}


\nsubsection{Comparação de ferramentas de sandboxing em sistemas GNU/Linux}

\begin{itemize}
	\item Alunos: Ellian Aragão e João Rezende
\end{itemize}
Sandboxing se refere ao isolamento de programas com objetivo de prevenir corrupção do espaço de usuário ou kernel e vulnerabilidades. Diversos mecanismos do sistema GNU/Linux podem ser utilizadas para este propósito, tais como: 
\begin{itemize}
	\item Firejail;
	\item Ferramentas de virtualização;
	\item Apparmor;
	\item Seccomp;
	\item \ldots
\end{itemize}



O objetivo do trabalho é comparar essas ferramentas de acordo com vários critérios relevantes para a segurança de sistemas operacionais.

\nsubsection{Aplicação de técnicas de SQL injection e suas contramedidas: um estudo de caso}

\begin{itemize}
	\item Alunos: Emanuelly Parreira da Silva e Lucas Bonfim Fernandes
\end{itemize}


Este trabalho consiste no estudo de caso sobre as técnicas de SQL injection {\it inband} e contramedidas para evitá-las. 



\subsubsection*{Referências}
\begin{itemize}
	\item \bibentry{alwan2017detection}	
	\item \bibentry{marashdeh2021survey}
\end{itemize}

\nsubsection{Análise comparativa dos provedores de computação em nuvem sob a ótica de virtualização e segurança}

\begin{itemize}
	\item Aluno: Matheus Fernandes Bezerra
\end{itemize}


Este trabalho visa fornecer uma análise comparativa sobre os provedores de computação em nuvem sob a ótica de tecnologias de virtualização e segurança, destacando as especificidades, pontos fortes e fracos de cada provedor.




\subsubsection*{Referências}
\begin{itemize}
	\item \bibentry{DBLP:journals/corr/abs-2208-14482}	
	\item \bibentry{li2011comparing}
	\item \bibentry{saraswat2020cloud}
\end{itemize}

\newpage
\section{Orientações concluídas}

\nsubsection{Construção paralela de $K^2$-trees utilizando GPUs}

\begin{itemize}
	\item Aluno: Rafael de Paula Filgueiras
\end{itemize}


\nsubsection{Uma proposta de Range min-Max tree k-ária para consultas sobre
árvores sucintas}

\begin{itemize}
	\item Aluna: Danyelle da Silva Oliveira Angelo
\end{itemize}

\nsubsection{Otimização e paralelização da construção de $k^2$-trees}

\begin{itemize}
	\item Aluna: Lauany Reis da Silva
\end{itemize}

\nsubsection{Projeto e implementação de consultas de ranqueamento e seleção em gramáticas baseadas em ordenação de sufixos por indução}

\begin{itemize}
	\item Aluno: Alan Cardoso Ferreira
\end{itemize}

\nsubsection{Investigação da metaheurística das aranhas sociais para o problema da cobertura de vértices}

\begin{itemize}
	\item Aluno: Rafael Araujo Gomes da Silva
\end{itemize}

\nsubsection{Sistema inteligente de divulgação de informações do IFG-Formosa}

\begin{itemize}
	\item Aluno: Matheus de Carvalho Sobrinho
\end{itemize}


\end{document}
